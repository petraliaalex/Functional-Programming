\documentclass{article}
\usepackage[utf8]{inputenc}
\usepackage{listings}
\usepackage{amssymb}
\usepackage{amsmath}
\usepackage{amsmath,amssymb,amsthm}
\renewcommand{\qedsymbol}{$\blacksquare$}

\title{Homework 3 - CSCI 305}
\author{Alex Petralia}
\date{February 2019}

\begin{document}



\maketitle
\section{Question 1: part a}
\\
\\
\begin{lstlisting}
M-ARY-Parent(int i, int m){
    return -> floor[\frac{(i - 2)}{m} + 1]
}

M-ARY-Child(int i, int m, int j){
    return -> m(i - 1) + j + 1
}
\end{lstlisting}




%---------------------------------------------------------------------------
\section{Question 1: part b}

Let $h$ be the height of the $m$-array heap.
\\Suppose the tree for the m-array heap is complete.
\\Complete meaning that all levels besides possibly the last level
\\are filled and all the nodes are as far left as they can be.
\\
\\Then by this logic we know: $1 + m + m^2 + . . . + m^h = n$.
\\
\\Thus $\frac{m^{h+1} - 1}{d - 1} = n$.
\\
\\Consequently $h = log_{m}(mn - 1 + 1) - 1 = log_{m}(mn) - 1$.
\\
\\We can further simply this as so:
\begin{align}
    \nonumber h = log_{m}(mn - 1 + 1) - 1 &= log_{m}(mn) - 1\\
    \nonumber &= log_{m}(n) + log_{m}(m) - 1\\
    \nonumber &= log_{m}(n) + 1 - 1\\
    \nonumber &= log_{m}(n)
\end{align}
\\Therefore the solution is $h = log_{m}(n)$ which can be equated to 
\\$\theta(log_{m}(n))$ to deal with the case when the heap isn't balanced.
 
%---------------------------------------------------------------------------




%---------------------------------------------------------------------------
\section{Question 1: part c}
\begin{lstlisting}
Extract-max(int[] A, int n, int m) { 
    int max = A[0];                                     
    A[0] = A[n - 1]; //copy last key to the root        
    n = n - 1; //decrease heap size                     
    Max-M-Heapify(A, n, 0, m);                          
    return max;                                         
}

//Max-M-Heapify is a modified version of maxHeapify using 
//functions we created in part a with an added parameter being m.

Max-M-Heapify(int[] A, int n, int index, int m){
    int[] childArray;                                                       
    childArray.length  = m + 1;                                             
    while(True){                                                            
        for(int i = 1; i < = m; i++){                                       
            //if the child is a leaf                                        
            if(childArray[i] == m * index + i && childArray[i] < len){      
                childArray[i] = -1 // -1 indicates the child is a leaf      
            }                                                               
        }                                                                   
        int maxChild = -1;                                                  
        int maxChildIndex;                                                  
        for(int j = 1; j <= m; j++){                                        
            if (M-ARY-Child(i, m) != -1 && A[M-ARY-Child(i, m)] > maxChild){
                maxChildIndex = M-ARY-Child(i, m);                           
                maxChild = A[M-ARY-Child(i, m);                             
            } 
        }
        if (maxChild == -1){                                                
            break;                                                          
        }
        if (A[index] < A[maxChildIndex]){                                   
            swap(A[index], A[maxChildIndex]);                               
            index = maxChildIndex;                                          
        }
    }
}
\end{lstlisting}
%Extract-max Analysis
%--------------------
We know the normal maxHeapify() run-time to be $\theta(log(n))$ which is 
\\derived from $T(n) = T(2n/3) + O(1)$ as we know the algorithm
\\recursively takes 2/3 of the array at each recursive step.
\\For Max-M-Heapify() the function recursively calls itself for m children
\\and knowing maxHeapify() in the worst case is $\theta(log(n))$ we can deduce that 
\\Max-M-Heapify() run-time is $mlog_{m}(n)$ where we factor in the
\\number of recursive calls being m.
\\
\\Also since Extract-max() just calls Max-M-Heapify() once we can 
\\conclude that it's run-time is $mlog_{m}(n)$.
\\
\section{Question 1: part d}
\begin{lstlisting}
Insert(int[] A, int key){                               
    A.length = A.length + 1;                            
    A[A.length] = (large negative number, eg -\infty)   
    A[A.length] = re_Heapify(A, A.length, key);         
}

//re-Heapify() fixes the heap after inserting a key
//into the heap.

reHeapify(int[] A, int i, int key){
    if(key < A[i]){
        break; //error: new key was smaller than current key
    }
    A[i] = key;
    while(i >1 && A[i] > A[M-ARY-Parent(i, m)]){
        swap(A[i], A[M-ARY-Parent(i, m)]);
        i = M-ARY-Parent(i, m);
    }
}
\end{lstlisting}
%Analysis of Insert():
%-----------------------
For reHeapify() we know for a m-array heap the max height will be
\\$log_{m}(n)$. Thus reHeapify() will run for a maximum of $log_{m}(n)$
\\times since at every iteration the node is moved a level up.
\\Therefore the time complexity for worst case is $\theta(log_{m}(n))$.

\section{Question 2}
\begin{lstlisting}
We are assuming the heap to be a min-heap.
keyFinder(int k, int index, int[] H, int m){
    if(index >= m){ //checks if the index exists
        return;
    }
    if(H[index] > k){
        \\skips the node and its descendents
        return; 
    }
    print(H[index]);
    keyFinder(k, left(index), H, m);
    keyFinder(k, right(index), H, m);
    //left() and right() will grab the index of the
    //left or right child of the node
}
\end{lstlisting}
By definition of a min-heap all of the descendants will also have values
\\greater than or equal to k if this is already the case with the current
\\node(identified with the given index). Thus the algorithm doesn't have 
\\to explore any further making the run time $O(k)$ since we skip all the
\\nodes that don't satisfy the condition for k.


~\\
\section{Question 3: part a}
1) \underline{The Algorithm} :
\begin{lstlisting}
findMe(int[] A, int l, int r, int z){
    int j;                              
    int m = floor[(l + r)/2];           
    if(A[m] == z){                      
        return m;                       
    }else if(m == l){                   
        return r;                       
    }else if(m == r){                   
        return l;                       
    }else if(A[m] > z){                 
        r = m;                          
        j = findMe(A, l, r, z);     
    }else if(A[m] < z){                 
        l = m;                          
        j = findMe(A, l, r, z);         
    }
    return j;                           
}



\end{lstlisting}
2) \underline{The Recurrence Relation (Worst Case)}:
\\
\\Comparisons of elements in the array are invoked on lines c3, c9 and c12.
\\In the worst case, the array gets divided in half recursively until an array of size 2 is what's left before
returning the final j value (the resulting index). We also know there will be $log(n)$ comparisons which is dependent on $T(n/2)$ recursive calls to find what $T(n)$ is.
\\Therefore we can solve the recurrence relation for the worst case as so:
\\
\begin{align}
    \nonumber T(n) &= T(\frac{n}{2}) + lg(n)\\
    \nonumber &= T(\frac{n}{4}) + lg(\frac{n}{2}) + lg(n)\\
    \nonumber &= T(\frac{n}{8}) + lg(\frac{n}{4}) + lg(\frac{n}{2}) +lg(n)\\
    \nonumber &= .  .  .
\end{align}
\\
After $log(n)$ entries we get: $log(n) + log(\frac{n}{2}) + ... + log(n/2^{log(n)})$.
\\
\\We can rewrite this as: 
\\$log(n) + (log(n) - 1) + (log(n) - 2) + (log(n) - 2) + (log(n) - log(n)) $.
\\
\\
\\Which is equivalent to this sum:
\\$log(n) + ... + 3 + 2 + 1 + 1$.
\\
\\Therefore we can a closed form of this sum which is $\frac{(log(n))(log(n) + 1)}{2} = \theta(log^{2}n)$.
\\\\\\\
3)\underline{Proving the algorithm(by substitution)}
\\
\\
\begin{proof}\\
\\
\\We first re-write the recurrence as: $T(2^{k}) &= T(2^{k - 1}) + k$.
\\Then we will define a new recurrence as $S(k) = S(k - 1) + k $.
\begin{align}
    \nonumber S(k) &= S(k - 1) + k  \\
    \nonumber &= S(k - 2) + S(k - 1) + k \\
    \nonumber &= S(k - 3) + S(k - 2) + S(k - 1) k\\
    \nonumber &= .  .  . \\
    \nonumber &= S(0) + 1 + 2 + 3 +  ... + k\\
    \nonumber &= S(0) + \theta(k^{2})
\end{align}
\\So if we assume $S(0) = 1$ then the recurrence will equate to $\theta(k^{2})$.
\\We also know $S(k) = T(2^{k}) = T(n)$ thus we get:
\\$T(n) - \theta(k^{2}) = \theta(log^{2}n)$.
\end{proof}



\end{document}
